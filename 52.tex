\section{
	Арифметические функции...  % не заменять на ldots
}

\textbf{Примеры. Функции Дирихле. Свёртка Дирихле и её кольцевые свойства. Мультипликативные функции. Примеры.}


В теории чисел часто встречаются функции от натурального параметра $n$, которые выдают некоторое число, которое как-то завязано на свойствах кольца $\mb Z/n$. Это несколько загадочная часть теории чисел. Эта область выглядит скорее как техническое средство. Тем не менее обходить её стороной не стоит. Основной тип вопросов, которые будут рассматриваться -- это вопросы асимптотического поведения этих функций, что может понадобится при оценке сложности теоретико-числовых алгоритмов.

\dfn[Арифметические функции] Пусть $R$ --- кольцо. Арифметической функцией со значением в $R$ называется отображение $f\colon \mb N \to R$.
\edfn

Обычно в качестве $R$ берут комплексные числа. Однако, иногда бывает полезно взять в качестве $R$ кольцо каких-нибудь функций.

\exm
\begin{enumerate}
\item $1(n)=1$.
\item $e(n)=\begin{cases} 1, n=1 \\ 0, \text{ иначе}\end{cases}.$
\item $I_k(n)=n^k$.
\item $\varphi(n)$ --- функция Эйлера.
\item И вообще, $f(n)=|V_{g_1,\dots,g_n}(\mb Z/n)|$, где $g_1,\dots,g_n$ какие-то целочисленные многочлены от одинакового числа переменных.
\item $\sigma(n)=\sum_{d|n}d$
\item Более общо $\sigma_k(n)=\sum_{d|n}d^k$.
\item В частности, $\sigma_0(n)=d(n)=\sum_{d|n} 1$ то есть число делителей.
\item $r(n)=|\{(x,y)\in \mb Z^2\,|\, x^2+y^2=n\}|$.
\end{enumerate}

Для последовательностей, удовлетворяющих линейным рекуррентным соотношением естественно в качестве производящей функции было взять степенную производящую функцию, так как они ведут себя как геометрическая прогрессия, то с арифметическими функциями дело обстоит иначе. Их величина обычно ограничена полиномом, а то и логарифмом $n$. Следовательно для их исследований пригодны другие производящие функции.

\dfn Пусть $a\colon \mb N\to \mb C$ --- арифметическая функция. Производящей функцией Дирихле или рядом Дирихле по $s$ называется следующее выражение
$$L(s)=\sum_{i=1}^{\infty} \frac{a(n)}{n^s}.$$ 
\edfn

\dfn В частности знаменитая дзета-функция Римана есть функция Дирихле для $1(n)$ то есть
$$\zeta(s)=\sum_{n=1}^{\infty}\frac{1}{n^s}.$$
\edfn

\rm Если последовательность $a(n)$ есть $O(n^{\alpha})$, то ряд Дирихле абсолютно сходится при всех вещественных $s>\alpha+1$, а на самом деле и при всех комплексных $s$ c $\re s>\alpha+1$.
\erm

\fct Пусть $L_1(s)$ -- ряд Дирихле для $a$, а $L_2(s)$ -- ряд Дирихле для $b$. Если ряды $L_1$ и $L_2$ сходятся для всех $s>s_0$ и в этой области верно равенство $L_1=L_2$, то функции $a$ и $b$ равны.
\efct

Последние два факта говорят нам, что про ряды Дирихле действительно можно думать, как про честные функции комплексного аргумента. Однако нас больше будет интересовать формальная сторона дела. Тем не менее указанная картинка будет мотивировать нас к некоторым определениям.

Представим себе два ряда Дирихле $L_1(s)=\frac{a_1}{1^s}+\frac{a_2}{2^s}+\dots$ и $L_2=\frac{b_1}{1^s}+\frac{b_2}{2^s}+\dots$. Перемножим их. Получится выражение вида 
$$\sum_{n,m} \frac{a_nb_m}{(nm)^s}=\sum \frac{1}{n^s}\sum_{d|n}a_db_{\frac{n}{d}}.$$
Это даёт нам основание ввести на всех арифметических функциях структуру умножения не с помощью покомпонентного произведения, а при помощи полученной формулы.

\dfn[Свёртка Дирихле] Пусть $a,b\colon \mb N \to R$ -- арифметические функции. Определим их свёртку Дирихле при помощи формулы $$c_n=\sum_{d|n}a_db_{\frac{n}{d}}=\sum_{d_1d_2=n}a_{d_1}b_{d_2}.$$
\edfn

\lm Относительно свёртки Дирихле и поточечного сложения арифметические функции образуют кольцо. Единицей кольца является функция $e$.
\elm

Мы сконцентрируемся сейчас на функциях специального вида.

\dfn Арифметическая функция $f$ называется мультипликативной, если $f(1)=1$ и для любых двух взаимно простых $(n,m)=1$ 
$$f(nm)=f(n)f(m).$$ 
\edfn
Возможно вы удивитесь такому определению. Можно было бы ожидать каких-то других слов. Однако именно такая <<неполная>> мультипликативность встречается очень часто.

\exm\\
Функции примеров 1),2),3),4),5) очевидно или по Китайской теореме об остатках являются мультипликативными. На самом деле функции примеров 6),7),8) и, если чуть-чуть подправить, то 9) так же являются мультипликативными. Попробуем пояснить эти факты.

