\section{
 Лемма Гаусса. Содержание многочлена... %не надо писать формулы и \ldots
}

\textbf{Лемма Гаусса. Содержание многочлена. Делимость в $Q(R)[x]$ и в $R[x]$.}

{\it Шпаргалка:}\\
Лемма: Пусть нет, возьмём $\min a_i, b_j \ndi p$, тогда $c_{i + j} \ndi p$. Следствие: поделим на $\cnt g, h$, убедимся что $\cnt f = 1$. Лемма про $Q(R)[x]$: $d_1, d_2$~--- НОК знаменателей, $c = \frac{d_1}{d_2}$.

\dfn
Область целостности $R$ назывется {\bf факториальным кольцом}, если для любого $a \neq 0 \in R$ существует представление $a$ в виде $a = \varepsilon p_1 \dots p_n$, где $\varepsilon \in R^*$, а $p_i$~--- простые.\\
Причём такое представление единственно с точностью до ассоциированности и перестановки
сомножителей. То есть для любого другого разложения $a = \varepsilon q_1 \dots q_m$ верно, что $n = m$ и существует перестановка $\sigma \in S_n q_i \sim p_{\sigma(i)}$
\edfn

\lm[Гаусс] Пусть $R$ -- факториальное кольцо. Тогда любой простой элемент $p$ из $R$ остаётся простым в $R[x]$.
\proof
Их будет два:
\begin{itemize}
	\item Теоретическое\\
	Элемент $p$~--- простой, если идеал $(p)$ в $R[x]$ простой, т. е. $R[x]/(p)$~--- область целостности.\\
	$(p)$~--- идеал вида $pR[x]$, т. е. он состоит из многочленов, все элементы которых кратны $p$. Тогда понятно, что $R[x]/(p) = R/p[x]$. $R/p$~--- область целостности, ну а колько многочленов над областью целостности тоже область целостности.
	\item И практическое\\
	Формально нам надо доказать, что если произведение двух многочленов $f(x)g(x)$ делится на $p$ (то есть все коэффициенты кратны $p$), то тогда какой-то из них делится на $p$. Пусть это не так. Возьмём тогда у $f$ и у $g$ самые младшие коэффициенты $a_i$ и  $b_j$, которые не делятся на $p$. Тогда посмотрим на коэффициент с номером $i+j$  в произведении. Он имеет вид $c_{i+j}= a_ib_j + \sum_{k \neq i} a_k b_{i+j -k}$. Поймём, что $c_{i+j}$ не делится на $p$: $a_i b_j$ не делится на $p$, а любое слагаемое в сумме делится, так как либо $k<i$ и тогда $a_i \di p$, либо $k>i$, тогда $i+j-k<j$ и $b_j \di p$. Противоречие.   
\end{itemize}
\endproof
\elm

\dfn Пусть $f(x)$ -- многочлен над факториальным кольцом $R$. Тогда содержанием $f$ называется $\cnt(f)=\Nod (a_i)$, где $a_i$ коэффициенты $f$.\\
(Помним, что НОД определён с точностью до ассоциированности.)
\edfn

\crl (из Леммы Гаусса)\\
 Если $f(x)=g(x)h(x)$, где $f,g,h \in R[x]$, то $\cnt(f)=\cnt(g)\cnt(h)$
\proof Для начала, упростим задачу, то есть сведём задачу к случаю $\cnt g= \cnt h =1$. Для этого надо рассмотреть многочлены $\frac{g}{\cnt g}$ и $\frac{h}{\cnt h}$. Их произведение есть $\frac{f}{\cnt{g}\cnt{h}}$ имеет содержание $\frac{\cnt f}{\cnt g \cnt h}$ и если показать его единичность, то мы добьёмся требуемого. Итак считаем, что $\cnt g= \cnt h=1$. Если $\cnt f$ не обратим, то $\cnt f \di p$, где $p$ простой элемент из $R$. Но тогда один из $g$ или $h$ делится на $p$ благодаря его простоте. 
\endproof
\ecrl

Вспомним, что для полей мы знаем про однозначность разложения. Так что внимальтельнее посмотрим на поле частных $Q(R)$.

\lm Пусть для многочлена $f(x) \in R[x]$  имеет место разложение $f(x)=g(x)h(x)$, где  $g(x)h(x) \in Q(R)[x]$. Тогда существуют такая константа $c \in Q(R)$, что $cg \in R[x]$ и $c^{-1}h \in R[x]$. Так что $f(x)=(cg(x))(c^{-1}h(x))$ -- есть произведение двух многочленов из $R[x]$, пропорциональных исходным.
\proof
Рассмотрим несократимую запись для коэффициентов $g$ и $h$ и НОК их знаменателей за $d_1$ и $d_2$ соответственно. Тогда $d_1g$ и $d_2h$ лежат в $R[x]$. Получаем равенство $d_1 d_2 f = (d_1 g)(d_2 h)$. Посчитаем содеражание слева и справа: $d_1 d_2\cnt f = \cnt(d_1 g) \cnt(d_2h)$. Таким образом, правая часть делится на $d_1d_2$. Вспомним как мы выбирали $d_1$ и поймём, что $(\cnt(d_1 g), d_1) = 1$ (аналогично про $d_2$). Значит $\cnt(d_1 g) \di d_2$ и $\cnt(d_2h) \di d_1$. Возьмём теперь $c= \frac{d_1}{d_2}$ и победим.

\elm