\section{
 След, его нетривиальность. Алгоритм... %не надо писать сюда мат формулы
}

\textbf{След, его нетривиальность. Ещё один алгоритм разложения для $q = 2$}.

Осталось единственное <<Но>>. Это не работает в характеристике 2. В случае характеристики 2 рассмотрим следующее выражение:
$$U(y)=y+y^q+\dots+y^{q^{d-1}}.$$
Это линейное отображение $\mb F_{q^d}\times \dots \times \mb F_{q^d} \to \mb F_{q}\times \dots \times \mb F_{q}$ лежащее внутри $R$. Я утверждаю, что это отображение невырождено. Для этого проверим это свойство для отображения $\mb F_{q^d} \to \mb F_q$. Для этого заметим, что $U(x)$ есть многочлен степени $q^{d-1}$ и, следовательно, не может иметь более $q^{d-1}$ корней. Это означает, что его ядро есть не всё $\mb F_{q^d}$, откуда получаем требуемое.

Пусть $q=2$. Возьмём теперь степени образующей $1,x,\dots, x^{2d-1}$. По китайской теореме об остатках существует такой многочлен $p(x)$, что $\deg p(x)<2d$ и  остаток $U(p(x))$ по модулю $f_1$ равен 0 и по модулю $f_2$ равен $1$. Но тогда есть и для некоторого $l$  $x^l$ обладает свойством, что $U(x^l)$ имеет разные остатки по модулю $f_1$ и $f_2$. Но тогда, один из этих остатков 0, а ещё один единица. Отсюда получаем, что $U(x^{l})$ -- даёт нетривиальный делитель нуля в $R$.

\rm Есть алгоритм, позволяющий свести разложение многочлена над $\mb F_{q^d}$ к разложению многочлена над $\mb F_q$.
\erm
