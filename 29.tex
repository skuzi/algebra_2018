\texttt{\texttt{•}}\section{
 Редукционный признак неприводимости. Примеры. Признак Эйзенштейна.
}
{\it Шпаргалка.}\\
 1) $a_n \ndi p$, f - неприводим в $R/p[x]$ $\Rightarrow$ неприводим над $Q(R)$. $cont = 1$ и неприводимость над $Q(R)$ $\Rightarrow$ неприводимость над $R$. 2) $a_n \ndi p$, все $a_i \di p$ $i<n$, но $a_0\ndi p^2$, то многочлен $f(x)$ неприводим. Пусть $b_0 \ndi p$.
\thrm[Редукционный критерий] Пусть $R$ факториальное кольцо, $f \in  R[x]$ многочлен, а $p$ -- простой элемент. Тогда, если старший коэффициент $f$ не делится на $p$ и $\ovl{f}$ неприводим в кольце $R/p[x]$, то он неприводим над $Q(R)$. 
\proof Хотим пользоваться тем, что при каких-то условиях неприводимость над $R$ эквивалентна неприводимости над $Q(R)$. Стрелка в правую сторону очевидна.
Можно поделить $f$ на $cont(f)$, чтобы воспользоваться знаниями из предыдущего билета о том, что неприводимые многочлены над $Q(R)$, чье содержание равно единице, - это все неприводимые многочлены в $R[x]$(так было на лекции). Кажется, что проще воспользоваться леммой из предыдущего билета о том, что если есть разложение
на два многочлена из $Q(R)[x]$, то есть разложение и на два многочлена из $R[x]$.
 Теперь можно доказывать теорему для $R$. Пусть $f=gh$, где $g,h$ --  не константы. Старшие коэффициенты $g$ и $h$ тоже не делятся на $p$. Имеем $\ovl{f}= \ovl{g}\ovl{h}$ и $\deg g = \deg \ovl{g}$ и $\deg h = \deg \ovl{h}$, что даёт нетривиальное разложение $\ovl{f}$ и приводит к противоречию.
\endproof
\ethrm

Вот примеры о том, как пользоваться этим критерием и что не надо забывать про условие со старшим коэффициентом. 

\exm\\
1) Многочлен $x^3+x+1$ неприводим над $\mb F_2=\mb Z/2$, потому что у него нет корней. Следовательно многочлены $3x^3+8x^2+5x+7$ и скажем, $5x^3-4x^2+x+15$ неприводимы над $\mb Q$.\\
2) Рассмотрим многочлен $px^2+x$. Он приводим, но по модулю $p$ -- неприводим.\\
3) Критерий из теоремы сформулирован не в самом сильном виде. А именно, представим себе, например, что по модулю 2 многочлен степени пять разложился в произведение двух неприводимых степени 2 и 3, а по модулю 3 -- в виде произведения степени 4 и 1. Ясно, что он неприводим.\\
4) Не стоит забывать, что если многочлен неприводим над $\mb R$, то он так же неприводим над $\mb Q$. Это, правда, очень слабый критерий, но в комбинации с пунктом 3) может что-то дать.\\



Есть, однако, такие многочлены, которые неприводимы, но раскладываются по модулю любого простого. Например, $$x^4+1=(x-e^{\frac{i\pi}{8}})(x-e^{\frac{3i\pi}{8}})(x-e^{\frac{5i\pi}{8}})(x-e^{\frac{7i\pi}{8}})= (x^2+i)(x^2-i)=(x^2+\sqrt{2}x+1)(x^2-\sqrt{2}x+1)=(x^{2}+\sqrt{-2}x+1)(x^{2}-\sqrt{-2}x+1).$$ Он не имеет корней, а любые множители степени 2 имеют нерациональный коэффициент.

С другой стороны по любому простому модулю либо из $-1$, либо из $2$ либо из $-2$ извлекается корень.\\
Комментарии к последнему примеру: на лекциях его не было, и последнее утверждение не за одну секунду доказывается.


\thrm[Признак Эйзенштейна] Пусть $R$ -- факториальное кольцо и $f(x)= a_0 + \dots + a_n x^n$. Если $a_n \ndi p$, все $a_i \di p$ $i<n$, но $a_0\ndi p^2$, то многочлен $f(x)$ неприводим.
\proof
Предположим, что $f=gh$. Заметим, что старшие коэффициенты $b_k$ и $c_l$ у $g$ и $h$ не делятся на $p$. Пусть так же $b_0 \ndi p$ у многочлена $g$. Рассмотрим самый младший коэффициент у $h$ не делящийся на $p$. Такой есть, потому что $c_l\ndi p$ и это точно не $c_0$. Пусть это $c_s$. Тогда $a_s=c_sb_0+c_{s-1}b_1+\dots+c_0b_s$. Все слагаемые, кроме первого делятся на $p$. Но тогда $a_s \ndi p$, что невозможно.
\endproof
\ethrm 