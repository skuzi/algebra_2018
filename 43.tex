\section{
 Лемма про производную. Лемма про корень... %не надо писать формулы и ldots
}

\textbf{Лемма про производную. Лемма про эффективное извлечение корня степени $p$.}

\textbf{Необработанная версия из конспекта Константина Михайловича}


Я опишу здесь некоторый набор соображений и алгоритмов касательно разложения многочленов на множители над конечным полем. 

Мы помним, что над полями характеристики 0 всегда легко выделить все  кратные множители многочлена просто взяв отношение $f$ и $\Nod(f,f')$. Однако, над конечными полями всё немного не так. Точнее

\lm Пусть $f= \prod g_i^{n_i}$. Тогда $$\Nod(f,f')=\prod_{n_i \ndi p}g_i^{n_i-1} \prod_{n_i \di p}g_i^{n_i}.$$
\proof
Рассмотрим неприводимый множитель $g_i$. Пусть $f(x)=g_i(x)^{n_i}g(x)$. Продифференцируем. Имеем $f'(x)= n_ig_i'(x)g_i^{n_i-1}+ g_i^{n_i}g'(x)$. Если $n_i\ndi p$, то кратность вхождения $g_i(x)$ в $f'(x)$ равна $n_i-1$. Если же $n_i\di p$, то $f'(x)=g_i^{n_i}g'(x)$, что показывает, что степень вхождения $g_i$ в $\Nod(f,f')$ не менее $n_i$. Но степень вхождения $g_i$ в $f(x)$ ровно $n_i$. Откуда степень вхождения $g_i$ в НОД $n_i$, что и требовалось.  
\elm

\lm Многочлен $h$ над конечным полем характеристики $p$ имеет нулевую производную тогда и только тогда, когда он является $p$-ой степенью. Извлечение степени можно провести эффективно.
\proof Как мы уже знаем с прошлого семестра, если $h'=0$, то $h=g(x^p)$. Посмотрим на коэффициенты $g$ -- $a_0, \dots, a_l\in \mb F_q$. Вспомним, что эндоморфизм Фробениуса $\Frob \colon \mb F_q \to  \mb F_q $ -- биекция. Иными словами из каждого элемента можно извлечь корень степени $p$. Пусть $b_i^p=a_i$. Тогда $f=b_0+b_1x+\dots+b_lx^l$ обладает свойством $f^p=g(x^p)=h$. Как вычислить корень степени $p$ из элемента? Для этого заметим, что обратное отображение к $\Frob \colon \mb F_q \to \mb F_q$ это $\Frob^{\circ n-1}$. 
\endproof
\elm 