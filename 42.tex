\section{
 Расширения поля, неприводимые многочлены... %не надо писать сюда формулы и ldots
}

\textbf{Расширения поля $F_q$ . Неприводимые многочлены как делители $x^{q^d} - x$.}

\textbf{Необработанная версия из конспекта Константина Михайловича}

\thrm Все расширения поля $\mb F_q$, где $q=p^n$ имеют $q^m$ элементов. Два расширения $\mb F_q$ из $q^m$ элементов изоморфны между собой. Внутри поля $\mb F_{q^m}$ есть подполе $\mb F_{q^l}$ только если $l|m$.
\proof
Если $L$ расширение $\mb F_q$, то оно имеет $q^{[L:\mb F_q]}$ элементов. Покажем существование. Возьмём поле из $q^m=p^{nm}$ элементов и рассмотрим в нём подполе $\mb F_q$ из $q=p^n$ элементов. Такое есть по предыдущей теореме. Это и даёт необходимое расширение. 

Покажем единственность такого расширения с точностью до $\mb F_q$ изоморфизма. Основная сложность состоит в том, чтобы проследить за сохранением $\mb F_q$ коэффициентов. Итак, пусть $L_1$ и $L_2$ -- расширения $\mb F_q$ из $q^m$ элементов. Такие поля изоморфны над $\mb F_p$. Пусть $\ffi \colon L_1 \to L_2$ изоморфизм над $\mb F_p$. Вообще говоря он не обязан переводить элементы из $\mb F_q$ в себя же. Нам надо подправить его, чтобы он так делал. Для этого заметим, что $\ffi(\mb F_q)=\mb F_q$. Таким образом у нас возникает автоморфизм $\mb F_q \to \mb F_q$. Он имеет вид $\Frob^{\circ i}$. Тогда на всём поле $L_2$ рассмотрим автоморфизм $\ffi'=\Frob^{\circ -i}$. Тогда композиция $\ffi'\circ \ffi$ и есть подходящий изоморфизм.

Теперь рассмотрим поле из $q^m$ элементов. Тогда в нём есть подполе из $q^l$ элементов только если $nm \di nl$. Но это происходит только если $m \di l$. Такое подполе единственно и автоматически снабжается структурой $\mb F_q$ расширения, так как содержит образ последнего при его вложении в $\mb F_{q^m}$.
\endproof
\ethrm

Покажем одну полезную теорему про многочлены над конечным полем.

\thrm Пусть $f(x)$ -- это неприводимый многочлен из $\mb F_q[x]$. Тогда $x^{q^m}-x \di f(x)$ тогда и только тогда, когда $\deg f(x) | m$.
\proof Пусть $x^{q^m}-x$ делится на $f(x)$. Тогда в поле $\mb F_q^m$ многочлен $f(x)$ имеет корень $\alpha$ (на самом деле там лежат все его корни). Теперь $\mb F_q[\alpha]$ подполе $\mb F_{q^m}$. Но тогда $\deg f(x) = [\mb F_q[\alpha]: \mb F_q] \di m $. 

Обратно, пусть $k=\deg f(x) | m$. Тогда в $\mb F_q^m$ есть подполе $\mb F_{q^k}$. Но такое подполе изоморфно $\mb F_q[x]/f$ и имеет внутри корень $\alpha$ многочлена $f(x)$. Но тогда $f(x)$ и $x^{q^m}-x$ не взаимно просты, откуда следует, что $x^{q^m}-x \di f(x)$, благодаря неприводимости последнего. 
\endproof
\ethrm
