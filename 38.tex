\section{
 Построение при помощи циркуля и линейки. Пример неразрешимого построения.


$x$ -- построимо $\Rightarrow$ оно алгебраическое и лежит в расширении $L/\mb Q$ степени $2^n$. Докажем индукцией по числу построений, рассмотрим уравнение пересечения с новым объектом степени $2$. $\cos \frac{\pi}{9}$ -- корень уравнения $4x^3 - 3x = \frac{1}{2}$.\\ 

 Напомню, что при построении циркулем и линейкой можно поставить пару начальных точек (задать масштаб), соединять две построенные  точки прямой и строить окружность с центром в построенной точке и с расстоянием, равным расстоянию между уже построенными двумя точками. Точка построена, если она есть точка пересечения построенных прямых и окружностей. 

Вещественное число $x$ называется построимым, если стартуя с точек $(0,0)$ и $(1,0)$ можно построить отрезок $(x,0)$. 

\thrm Если вещественное число $x$ построимо, то оно алгебраическое и лежит в расширении $L/\mb Q$ степени $2^m$.
\proof Доказательство идёт индукцией по числу построений. Пусть уже построены прямые $l_i$ и окружности $O_j$. Заметим, что коэффициенты в уравнениях $O_i$ и $l_j$ по индукционному предположению лежат в подполе $L\subseteq \mb R$ степени $2^m$ над $\mb Q$. Тоже касается и новой прямой $l$ (или окружности $O$). Посмотрим на пересечение окружности $O_j$ и новой $l$. Она заданы уравнениями $(x-a)^2+(y-b)^2=r^2$ и $cx+dy=f$. Пусть $c\neq 0$. Тогда первое уравнение переписывается в виде $$(f-dy)^2+c^2(y-b)^2=c^2r^2$$
Его коэффициенты из $L$, а решение $y$ лежит либо в $L$ либо в расширении степени 2 над $L$. Случай пересечения двух окружностей сводится к пересечению окружности и прямой.  
\endproof
\ethrm

\crl
Нельзя разбить произвольный угол на три части при помощи циркуля и линейки.
\proof Например, угол $\frac{\pi}{9}$ нельзя. Действительно, построимость угла и его косинуса равносильны. Косинус $\frac{\pi}{9}$ удовлетворяет уравнению $4x^3-3x=\frac{1}{2}$. Это неприводимый над $\mb Q$ многочлен (так как его корни это $\cos (\alpha)$ : $\cos(3\alpha) = \frac{1}{2}$, то есть $\alpha = \frac{\pi}{9} + \frac{2\pi k}{3}$) степени 3 и его корни не могут лежать в расширении степени $2^m$. Следовательно, построение невозможно. 
\endproof
\ecrl

}
