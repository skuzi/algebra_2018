\section{
 Единственность тензорного произведения. Размерность тензорного произведения.
}

\textbf{Необработанная версия из конспекта Константина Михайловича}

\lm Если тензорное произведение существует, то оно единственно.
\proof Это типичное <<категорное>> доказательство. Пусть есть два пространства $W_1$ и $W_2$ вместе с отображениями $i_1 \colon V_1\times \dots\times V_n \to W_1$ и $i_2 \colon V_1\times \dots\times V_n \to W_2$, для которых выполнены аксиомы тензорного произведения. Заметим, что так как $i_2$ полилинейное, то есть такое $\hat{i_2} \colon W_1 \to W_2$, что $$\hat{i_2}\circ i_1= i_2.$$ 
Аналогично существует $\hat{i_1} \colon W_2 \to W_1$, что
$$\hat{i_1}\circ i_2= i_1.$$ 
Покажем, что $\hat{i_1} $ и $\hat{i_2}$ взаимно обратны. Действительно, имеем
$$\hat{i_1}\circ \hat{i_2}\circ i_1= \hat{i_1}\circ i_2= i_1,$$
что означает, что отображение $\hat{i_1}\circ \hat{i_2}$ есть то самое единственное отображение $W_1\to W_1$, которое гарантируется благодаря полилинейности $i_1$ и того, что само $W_1$ -- тензорное произведение. Но есть другой кандидат на эту роль -- это $\id_{W_1}$. По единственности 
$$\hat{i_1}\circ \hat{i_2}=\id_{W_1}.$$
Аналогично проверяется равенство для второй композиции.
\endproof
\elm

\dfn Будем обозначать элемент $i(v_1,\dots,v_n)=v_1\otimes \dots \otimes v_n$. 
\edfn

Теперь необходимо посчитать что-то про тензорное произведение. Например, научиться считать размерность тензорного произведения и находить его базис.
\thrm Пусть $e_{i1},\dots,e_{in}$ базис $V_i$. Тогда $e_{1j_1}\otimes \dots \otimes e_{nj_n}$ базис $V_1 \otimes \dots \otimes V_n$. В частности, 
$$\dim V_1 \otimes \dots \otimes V_n= \prod_{i=1}^n \dim V_i.$$ 
\proof Прежде всего заметим, что набор $e_{1j_1}\otimes \dots \otimes e_{nj_n}$ является порождающей системой для тензорного произведения. Далее, по определению тензорного произведения,
$$\Hom(V_1,\dots,V_n, K) \cong \Hom(V_1\otimes \dots \otimes V_n,K).$$
Размерность последнего пространства совпадает с размерностью $V_1\otimes \dots \otimes V_n$. С другой стороны, полилинейное отображение $h \in \Hom(V_1,\dots,V_n, K)$ однозначно задаётся $\prod_{i=1}^n \dim V_i$  параметрами $h(e_{1j_1}, \dots,e_{nj_n})$. Комбинируя эти два факта получаем, что размерность $\dim V_1 \otimes \dots \otimes V_n$ есть $\prod_{i=1}^n \dim V_i$. Отсюда, любая порождающая система такого размера есть базис. В частности, набор $e_{1j_1}\otimes \dots \otimes e_{nj_n}$.
\endproof
\ethrm

