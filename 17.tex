\section{
 Тензорное произведение... % не заменять на ldots
}

\textbf{Тензорное произведение линейных отображений. Кронекерово произведение. Тензорное произведение операторов и его собственные числа. Категорное произведение графов.}

\dfn
	{\bf Тензорное произведение линейных отображений.}

	Пусть есть набор линейных отображений $f_i:\ U_i\to V_i$. Определим отображение
	$$
	f_1\otimes\cdots\otimes f_n:\ \otimes U_i\to\otimes V_i
	$$
	по правилу
	$$
	(f_1\otimes\cdots\otimes f_n)(u_1\otimes\cdots\otimes u_n) = f_1(u_1)\otimes\cdots\otimes f_n(u_n).
	$$
\edfn

\lm
	Отображение с таким свойством единственно.
	\proof
		Таким отображением мы определили значения $(f_1\otimes \cdots\otimes f_n)$ на базисе $(\otimes U_i)$, так как базис состоит из какого-то подмножества тензорят, а отображение мы определили от каждого тензорёнка.
	\endproof
\elm

\lm Указанное отображение корректно задано.

	\proof

		Рассмотрим диаграмму:
\begin{center}
\begin{tikzpicture}
\node (A) at (0, 1) {$U_1\times\cdots\times U_k$};
\node (B) at (0, 0) {$U_1\otimes\cdots\otimes U_k$};
\node (C) at (4, 1) {$V_1\times\cdots\times V_k$};
\node (D) at (4, 0) {$V_1\otimes\cdots\otimes V_k$};
\path[->,font=\scriptsize,>=angle 60]
(A) edge node[above]{$(f_1,\cdots, f_k)$} (C)
(A) edge node[above]{} (B)
(C) edge node[right]{$T$} (D);
\path[->,font=\scriptsize,>=angle 45, dashed]
(B) edge node[below]{$f_1\otimes\cdots\otimes f_n (?)$}  (D);
\end{tikzpicture}
\end{center}

		Посмотрим на композицию $T\circ (f_1, \cdots, f_k)$. Она полилинейна, так как это композиция двух полилинейных отображений.

		Значит (по определению тензорного произведения) существует и единственно линейное отображение $f_1\otimes\cdots\otimes f_n$. 

		Заметим, что это отображение удовлетворяет нашему требованию $(f_1\otimes\cdots\otimes f_k)(u_1\otimes\cdots\otimes u_k) = f_1(u_1)\otimes\cdots\otimes f_k(u_k)$, так как для набора $(u_1,\cdots,u_k)$: $(f_1\otimes\cdots\otimes f_k)(u_1\otimes\cdots\otimes u_k) = T(f_1(u_1),\cdots f_k(u_k)) = f_1(u_1)\otimes\cdots\otimes f_k(u_k)$ 
	\endproof
\elm

Теперь хочется разобраться с тем, как выглядит матрица тензорного произведения линейных отображений.

\lm
	Пусть $L_1 \colon V_1 \to U_1$, а $L_2 \colon V_2 \to U_2$. Пусть $e_1,\dots, e_{n_1}$ базис $V_1$,  $e_1',\dots, e_{n_2}'$ базис $V_2$,  и $f_1,\dots, f_{m_1}$ -- базис $U_1$, а $f_1',\dots, f_{m_2}'$ -- базис $U_2$. 
	Упорядочим базисы тензорных произведений -- удобно это сделать, например, в лексикографическом порядке (номер первой координаты важнее).
	Тогда матрица  $L_1\otimes L_2$  разобьётся на $n_1m_1$ блоков в каждом из которых будет стоять $ A_{ij} B$, где $i,j$ -- номер блока, а $A$ и $B$ матрицы $L_1$ и $L_2$ соответственно.
	\proof
		По определению тензорного произведения отображений: $(L_1\otimes L_2)(e_i\otimes e_j') = L_1(e_i)\otimes L_2(e_j') = (Ae_i)\otimes (Be_j') = (\sum\limits_{k} A_{k, i} f_k)\otimes(\sum\limits_{l} B_{l, j} f_l')$

		Когда раскроем скобки, при $f_k\otimes f_l'$ будет коэффицент $A_{k, i}\cdot B_{l, j}$

		Получился столбец матрицы:

		$$\pmat
		A_{1, i}\cdot B_{1, j}\\
		A_{1, i}\cdot B_{2, j}\\
		\vdots\\
		A_{1, i}\cdot B_{m_2, j}\\
		\vdots\\
		A_{m_1, i}\cdot B_{m_2, j}
		\epmat$$

		Это столбец при $e_i\otimes e_j'$ той матрицы, что и хотели.
	\endproof
\elm

\dfn
	Такая матрица называется {\bf кронекеровым (или тензорным) произведением} матриц $A$ и $B$ и обозначается $A\otimes B$.
\edfn

\lm
	У оператора $A\otimes B$ собственные числа~--- это попарные произведения с.ч. для $A$ и $B$. 
	\proof 
		Действительно, рассмотрим жорданов базис для $A$ и $B$~--- $v_1,\dots,v_n$ и $u_1,\dots, u_m$. Тогда рассмотрев базис $v_i\otimes u_j$ заметим, что под диагональю будут стоять нули, а на диагонали~--- попарные произведения собственных чисел $A$ и $B$.
	\endproof
\elm

\dfn 
	Пусть $G$ и $H$~--- два графа(возможно ориентированных). Тогда их {\bf категорным произведением} называется граф чьи вершины есть пары вершин $G$ и $H$ и ребро между парами $(u_1,v_1)$ и $(u_2,v_2)$ проводится только если есть рёбра $u_1 \to u_2$ и $v_1 \to v_2$. {\bf Декартовым произведением} графов $G$ и $H$ называется граф на тех же вершинах с ребром между парами если $u_1=u_2$ и есть ребро $v_1\to v_2$ или, симметрично, $v_1=v_2$ и есть ребро $u_1 \to u_2$. Разумеется для неориентированных графов эта конструкция снова выдаёт неориентированный граф.

	Обозначается $G\times H$
\edfn


\lm
	Спектр категорного произведения графов состоит из всех возможных попарных произведений собственных чисел графов.
	\proof 
		Заметим, что матрица смежности категорного произведения графов~--- это тензорное произведение матриц смежности исходных графов.

		Более формально: $A(G_1\times G_2) = A(G_1\otimes G_2)$. Во втором случае единички стоят только там, где есть оба ребра. Этого и хотели.
	\endproof
\elm