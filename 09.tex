\section{
 Положительные матрицы. Теорема Перрона.
}

\textbf{Необработанная версия из конспекта Константина Михайловича}

\dfn
Для каждого графа $G$ можно построить  несколько  различных матриц, которые кодируют его структуру. Прежде всего это три квадратные матрицы  размера $n\times n$, где $n$ -- это число вершин $G$. 
Первая -- матрица смежности  $A(G)$, которая полностью определяет граф $G$
$$a_{ij}=\begin{cases} 1, \text{ если вершины $i$ и $j$ соединены ребром}\\
0, \text{ иначе }
\end{cases}.$$

Так же нам уже встречалась матрица случайного блуждания  $P(G)$

$$P_{ij}=\begin{cases}
\frac{1}{d_j}, \text{ если есть ребро $j\to i$}\\
1, \text{ если из вершины не исходит рёбер} \\
0, \text{ иначе }
\end{cases}.$$
Кроме того, полезна бывает матрица инцидентности $B(G)$ размера $n\times m$, где $m$ -- число рёбер.
\edfn

В прошлом семестре мы с вами поняли, что для понимания того, как устроен предел последовательности $P^nv$, необходимо представлять себе как устроены собственные числа матрицы $P$. Прежде всего мы с вами понимали, что у матрицы $P$ есть собственное число 1. Однако встаёт несколько вопросов:\\
1) Какова кратность единицы, как собственного числа?\\
2) Есть ли другие собственные числа, равные единице по модулю?\\
3) Если $Pv=v$, то мы хотели бы интерпретировать $v$ как вектор весов для вершин графа. Верно ли, что $v$ можно выбрать положительным? Сколько таких независимых $v$?

Понятно, что в общем случае ответ на первые два вопроса <<нет>>.

\exm \\
1) Рассмотрим граф 
\begin{center}
\begin{tikzpicture}

\begin{comment}
\draw [fill] (0,0) circle [radius=0.05];
\draw [fill] (1,0.5) circle [radius=0.05];
\draw [fill] (1,-0.5) circle [radius=0.05];
\end{comment}

\node (A) at (0,0) {3};
\node (B) at (1,0.5) {1};
\node (C) at (1,-0.5) {2};
\path[->,font=\scriptsize,>=angle 60]
(A) edge (B)
(A) edge (C);
\end{tikzpicture}
\end{center}
У его матрицы $P$ очевидно есть два собственных вектора $(1,0,0)$ и $(0,1,0)$ с собственным числом 1.\\
2) Рассмотрим граф $C_n$ -- ориентированный цикл длины $n$. Его спектр -- это корни степени $n$ из единицы.\\


Сейчас мы докажем, что при некоторых предположениях на матрицу для неё ответы на все три вопроса оказываются положительными. Эти предположения не будут выполнены для матриц $A(G)$ и  $P(G)$ непосредственно, однако мы тем не менее сможем извлечь пользу.

\dfn Назовём матрицу $A$ положительной (не путать с положительно определённым оператором), если все её элементы $A_{ij}$ строго положительны. Будем писать в этом случае $A>0$.
\edfn

\dfn Назовём матрицу  $A$ не отрицательной, если $A_{ij}\geq 0$. Обозначение $A \geq 0$.
\edfn

\thrm[Перрон, 1907] Если матрица $A$ положительна, то наибольшее по модулю собственное число $A$ единственное и является вещественным и положительным. Это собственное число не является кратным корнем характеристического многочлена. Собственный вектор для этого собственного числа положителен.
\ethrm
\proof Пусть $\lambda$ -- наибольшее по модулю собственное число и $Ax=\lambda x$. Можно считать, что $|\lambda|=1$. Тогда покажем, что $A|x|=|x|$.

Прежде всего мы имеем цепочку неравенств $|x|=|Ax|\leq |A||x|=A|x|$, где все неравенства подразумеваются покомпонентными. Обозначим за $z=A|x|$. Это вектор состоящий из положительных координат и рассмотрим вектор $y=z-x$. Вектор $y$ неотрицателен. При этом если $y=0$, то мы доказали то, что хотели. Предположим, что есть координата $y_i>0$. Тогда $Ay$ -- положительный вектор, то есть существует такое $\eps>0$, что $Ay>\eps z$. Распишем это неравенство: $Az - z= Az-A|x|> \eps z$ или же $\frac{A}{1+\eps}z>z$. Ввиду положительности правой и левой части мы без сомнений можем применить оператор $\frac{A^n}{(1+\eps)^n}$ к правой и левой части и получить верное неравенство. Итого имеем цепочку 
$$\frac{A^n}{(1+\eps)^n}z>\frac{A^{n-1}}{(1+\eps)^{n-1}}> \dots > z.$$
Но оператор $\frac{A}{1+\eps}$ имеет собственные числа по модулю меньшие 1 и поэтому, как мы знаем с прошлого семестра, предел левого выражения равен 0. Противоречие!

Итак, в частности, единица собственное число. Покажем теперь, что нет отличных от единицы собственных чисел. Пусть $\lambda$ собственное число $A$ с $|\lambda|=1$ и $x$ -- соответствующий собственный вектор. Тогда $A|x|=|x|=|Ax|$. Заметим, что все координаты $x$ отличны от нуля. Рассмотрим $i$-ую координату. Имеем $\sum A_{ij}|x_j|=x_i=|\sum A_{ij}x_j|$. Посмотрим на это равенство как на равенство нормы векторов в $\mb R^2$. Хорошо известно, что сумма норм больше или равна нормы суммы и равенство достигается тогда и только тогда, когда вектора сонаправлены. Итого координаты $x_i$ должны быть сонаправлены, но это означает, что $x=e^{i\ffi} |x|$ и следовательно $\lambda=1$. 

Покажем, что единица не кратный корень. Действительно, пусть есть два собственных вектора $x_1$ и $x_2$. Тогда подберём $c$, так что $x_1-cx_2$ имеет нулевую координату. Получаем противоречие, так как $|x_1-cx_2|$ неотрицательный вектор для 1, но при этом с нулевой координатой. Осталось разобрать случай, когда $x_2$ присоединён к $x_1>0$, то есть $Ax_2=x_2+x_1$. Тогда имеем $A^nx_2=x_2 +nx_1$. Это значит, что какие-то коэффициенты $A^n$ растут по крайней мере линейно по $n$. Но тогда и коэффициенты $A^nx_1=x_1$ тоже растут по крайней мере линейно, что очевидно не так.
\endproof
