\section{
 Максимум квадратичной формы на сфере. Теорема Куранта-Фишера.
}

TODO: шпаргалка\\
\\
Теперь обратимся к вопросам, связанным с вещественными самосопряжёнными операторами. Для этого заметим, что с каждым самосопряжённым оператором $L$ на евклидовом пространстве можно связать билинейную симметричную  форму $\lan x,Ly\ran$ или, что эквивалентно, квадратичную форму $\lan x,Lx\ran$. Безусловно по квадратичной форме можно обратно восстановить оператор. 

Рассмотрим один из вопросов, связанных с такой конструкцией, а именно, рассмотрим задачу о нахождении нормы линейного оператора $L \colon U \to V$ между двумя евклидовыми пространствами. Для того, чтобы найти  норму необходимо найти (по одному из определений нормы) $$\max_{x\neq 0}\sqrt{\frac{\lan Lx,Lx\ran}{\|x\|^2}}=\sqrt{\max_{x\neq 0}\frac{\lan L^*Lx,x\ran}{\|x\|^2}}=\sqrt{\max_{\|x\|=1} \lan L^*Lx,x\ran}.$$
Таким образом, нахождение нормы оператора свелось к задаче максимизации квадратичной формы на единичной сфере. Заметим, что максимум действительно достигается благодаря компактности сферы.

Оказывается, что довольно легко найти максимум или минимум квадратичной формы на сфере.



\thrm Пусть $V$ -- евклидово пространство, $A$ -- самосопряжённый оператор на $V$, а $q(x)=\lan x,Ax\ran$ -- соответствующая квадратичная форма. Тогда 
$$\max_{ x\in V } \frac{q(x)}{||x||^2}=\max_{\substack{ x\in V \\ ||x||=1}} q(x)=\lambda_1,$$
где $\lambda_1$ - наибольшее собственное число оператора $A$ и достигается на собственном векторе $v_1$, соответствующему $\lambda_1$. Аналогично минимум равен минимальному собственному числу $A$. 
\proof
Пусть $v=\sum c_i e_i$, причём $1=||v||^2=\sum c_i^2$. Тогда $\lan Av,v\ran = \sum c^2_i \lambda_i $, что меньше $\lan A e_1,e_1\ran= \lambda_1= \sum c_i^2 \lambda_1$ т.к. $\forall i, \lambda_i \le \lambda_1$ по выбору $\lambda_1$. Доказательство для минимума полностью аналогично.
\endproof
\ethrm

Эта теорема, кроме, собственно, решения задачи, даёт геометрическую характеризацию первого собственного числа. Вопрос: можно ли аналогично охарактеризовать другие собственные числа? Ответ получается не таким простым, но, тем не менее, полезным.

\thrm[Куранта-Фишера] Пусть $q(x)=\lan x, Ax\ran$. Тогда $k$-ое по убыванию собственное число $\lambda_k$ для $A$ есть 
$$ \lambda_k=\max_{\dim L=k} \min_{\substack{ x\in L \\ ||x||=1}} q(x) = \min_{\dim L=n-k+1} \max_{\substack{ x\in L \\ ||x||=1}} q(x).$$
Причем максимум достигается на инвариантном подпространстве, содержащем собственные вектора для $\lambda_1,\dots,\lambda_k$.
\ethrm
\proof Пусть $U$ --- подпространство на котором достигается максимум, причём допустим, что максимум больше $\lambda_k$. Тогда рассмотрим подпространство $W=\lan v_k,\dots,v_n\ran$, где $v_i$ --- собственный вектор соответствующий $i$-ому по убыванию собственному числу. Заметим, что $U\cap W=\{0\}$, так как на $W$ форма принимает значения меньше или равные $\lambda_k$ (смотри предыдущую теорему), а на $U$ -- строго большие. Однако $\dim W=n-k+1$. Приходим к противоречию с подсчётом размерности пересечения. \\
\\
Максимум достигается на подпространстве $w = \lan \lambda_1, \ldots, \lambda_k \ran$, смотри предыдущую теорему.\\
\\
Вторая формулировка данной теоремы доказывается полностью аналогично.
\endproof
