\section{
 Алгоритм Берлекэмпа.
}


\thrm[Алгоритмы Берлекэмпа] Пусть $f(x)\in \mb F_q$ без кратных множителей. Тогда существуют детерминированный полиномиальный по $n$ (но не по $\log q$)  алгоритм раскладывающий $f$ на множители. 
\proof Первое соображение, которое мы применим, будет состоять в том, что мы переформулируем  задачу факторизации многочлена $f$ как задачу про некоторое кольцо. Точнее, пусть $f=h_1\dots h_l$ разложение на неприводимые. Тогда по Китайской теореме об остатках 
$$R= \mb F_q[x]/f\cong \mb F_q[x]/h_1 \times \dots \times \mb F_q[x]/h_l.$$ 
Заметим, что нахождение нетривиального делителя нуля в $R$ равносильно нахождению делителя $f$. Заметим, что, в свою очередь, $\mb F_q[x]/h_i$ -- поле из $q^{\deg h_i}$ элементов. Делителем нуля является любой элемент с хотя бы одним нулём в компоненте.


В каждом таком поле есть единственное подполе из $q$ элементов, состоящее из решений уравнения $x^q-x=0$. Если рассмотреть множество решений этого уравнения в $R$, то оно будет состоять из $l$-ек покомпонентных решений. Иными словами множество решений уравнения $x^q-x$ в $R$ есть подалгебра $R'$, изоморфная $\mb F_q\times \dots \times \mb F_q$, взятое $l$ раз. Если мы найдём делитель нуля в этой подалгебре, то найдём и в исходной. Заметим, что удельно, делителей нуля в этой подалгебре больше чем в исходной. 

Как найти $R'$? Для этого надо найти все решения уравнения $x^q-x=0$ в $R$. Второе соображение состоит в том, что это уравнение линейно (над $\mb F_q$). Чтобы решить это линейное уравнение надо составить его матрицу. У отображения $x \to -x$ матрица $-E_n$, где $n=\deg f$. У оператора $x \to x^q$ матрица легко считается. Далее достаточно применить любой из методов для решения систем линейных уравнений. Заметим, что если алгебра $R'$ одномерна (она не менее чем одномерна, так как константа всегда решение), то многочлен $f$ неприводим.

Теперь мы нашли $R'$. Построим детерминированный алгоритм нахождения разложения. Напомню, что нам надо получить делитель нуля, то есть элемент, у которого хоть одна компонента не 0. Возьмём произвольный не константный элемент $h$ из $R'$.  Тогда $h$ соответствует $l$-ка $(a_1,\dots,a_{l})$.  Переберём все константы $c$ из $\mb F_q$. Их $q$ штук (это и даёт неполиномиальность алгоритма по $\log q$). Тогда $h-c$ для, например, $c=a_1$ есть нетривиальный делитель 0 (он не ноль, потому что $h$ не константа).

Делитель $f$ теперь можно найти как $\Nod(f,h-c)$.

\endproof
\ethrm

