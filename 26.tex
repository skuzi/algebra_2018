\section{ Определитель. Формула Бине-Коши.}

\textbf{Необработанная версия из конспекта Константина Михайловича}


Покажем способ применения внешней степени для доказательства тождеств про определители. Прежде всего пусть есть отображение $L \colon U \to V$, где $\dim U= \dim V = n$ и $A$ матрица $L$ в базисах $e_1,\dots e_n$ и $f_1,\dots,f_n$. Тогда $$\Lambda^n L(e_1\wedge \dots \wedge e_n) = \det A \,\,f_1 \wedge \dots \wedge f_n.$$
Из этого замечания уже легко получить мультипликативность определителя. Поступая аналогично можно доказать более общую теорему:

\dfn Пусть $A$ -- матрица из $M_{m\times n}(K)$. Тогда если $\Gamma \subseteq \{1,\dots,n\}$. Тогда за $A_{\Gamma}$ обозначим матрицу состоящую из столбцов матрицы $A$ с элементоами из $\Gamma$.

Аналогично, если $\Gamma \subseteq \{1,\dots,m\}$ то за $A^{\Gamma}$ обозначим подматрицу $A$, из строк, чьи индексы лежат в $\Gamma$.
\edfn

\thrm[Формула Бине-Коши] Рассмотрим две матрицы $A\in M_{m\times n}(K)$ и $B\in M_{n\times m}(K)$. Пусть $m\leq n$. Тогда
$$\det(AB)=\sum_{\substack{\Gamma \subseteq \{1,\dots,n\}\\ |\Gamma|=m}} \det A_{\Gamma} \det B^{\Gamma}.$$
\proof Рассмотрим линейный отображения заданные матрицей $B \colon K^m \to K^n$ и $A\colon K^n \to K^m$. Тогда $\det AB = \Lambda^m (AB)$ как операторы из $K \to K$. С другой стороны $\Lambda^m(AB)=\Lambda^m(A) \Lambda^m(B)$. Вычислим матрицы этих отображений. Матрица $\Lambda^m(B)$ есть столбец высоты $C_n^m$, чьи элементы проиндексированны $\Gamma \subseteq \{1,\dots,n\}$ размера $m$. Аналогично матрица $\Lambda^m(A)$  есть строчка, чьи элементы проиндексированны аналогично.
Пусть $e=e_1\wedge \dots \wedge e_m$ единственный базисный элемент $\Lambda^m(K)\cong K$, а $f_1,\dots,f_n$ -- стандартный базис $K^n$. Тогда 
$$\Lambda^m(B)e= \sum_{i_1,\dots,i_m} B_{i_1 1}\dots B_{i_m m} \,f_{i_1}\wedge \dots \wedge f_{i_m}=\sum_{\substack{\Gamma \subseteq \ovl{1,n} \\ |\Gamma|=m}}\,\,\,\, \sum_{\sigma\colon \ovl{1,m} \to \Gamma} B_{\sigma(1)1}\dots B_{\sigma(m)m}\,\sgn{\sigma}\,f_{\Gamma}.$$
Здесь под знаком $\sigma$ подразумевается знак перестановки, если перенумеровать элементы $\Gamma$ по порядку. Теперь заметим, что при $f_{\Gamma}$ коэффициент в точности $\det B^{\Gamma}$. С другой стороны, аналогично первому вычислению с определителем
$$\Lambda^m(f_{\Gamma})=\det(A_{\Gamma})e.$$
Осталось перемножить строчку на столбец.
\endproof
\ethrm
